\documentclass[11pt,a4paper]{moderncv}

\usepackage{enumitem}
% moderncv themes
%\moderncvtheme[blue]{casual}                 % optional argument are 'blue' (default), 'orange', 'red', 'green', 'grey' and 'roman' (for roman fonts, instead of sans serif fonts)
\moderncvtheme[blue]{classic}                % idem

\usepackage[T1]{fontenc}
\usepackage{lmodern}
% character encoding
\usepackage[utf8]{inputenc}                   % replace by the encoding you are using
\usepackage[english]{babel}

% adjust the page margins
\usepackage[scale=0.85]{geometry}
\recomputelengths                             % required when changes are made to page layout lengths

\fancyfoot{} % clear all footer fields
\fancyfoot[LE,RO]{\thepage}           % page number in "outer" position of footer line
\fancyfoot[RE,LO]{\footnotesize } % other info in "inner" position of footer line

%number is section
%\newcounter{secnumber}
%\renewcommand\sectionstyle[1]{{%
		%  \refstepcounter{secnumber}%
		%  \sectionfont
		%  \textcolor{color1}{\thesecnumber~#1}%
		%}}

% personal data
\firstname{\Huge Sim\'on Emmanuel}
\familyname{Guti\'errez Brida}
\title{Curriculum Vitae} 
%\address{<Address Street>}{<Addres Place>}
\mobile{+549 (358) 4396 444}
\phone{+549 (358) 4676 235}
%\fax{<Fax number>}
\email{sgutierrez@dc.exa.unrc.edu.ar}
\homepage{http://saiema.github.io/}
\social[orcid]{0000-0003-3115-2739}

%\extrainfo{additional information (optional)} % optional, remove the line if not wanted
%\photo[100pt]{pict.JPG}
%\photo[84pt]{foto.jpg}                         % '64pt' is the height the picture must be resized to and 'picture' is the name of the picture file; optional, remove the line if not wanted
%\quote{"Success is the ability to go from failure to failure without losing your enthusiasm." -- Winston Churchill}                 % optional, remove the line if not wanted

%\nopagenumbers{}                             % uncomment to suppress automatic page numbering for CVs longer than one page


%----------------------------------------------------------------------------------
%            content
%----------------------------------------------------------------------------------

\usepackage{amsmath}
\usepackage{amssymb}
\usepackage{latexsym}
\usepackage{wasysym}

\begin{document}
	\maketitle
	
	%Section
	\section*{Información Personal}
	%\cvcomputer{Nombre:}{Renzo Degiovanni}{D.N.I:}{32705627}
	%\cvline{}{Lugar de Nacimiento: R\'io Cuarto, C\'ordoba, Argentina}
	%\cvcomputer{Fecha y Lugar de Nacimiento:}{10/04/1987}{Estado Civil:}{Uni\'on de Hecho}
	%\cvcomputer{Lugar de Nacimiento:}{R\'io Cuarto, C\'ordoba, Argentina}{N\'umeros de Hijos:}{1}
	%\cvcomputer{Fecha de Nacimiento:}{10/04/1987}{Lugar de Nacimiento:}{R\'io Cuarto, C\'ordoba, Argentina}
	%\cvcomputer{Estado Civil:}{Uni\'on de Hecho}{Hijos:}{1}
	%\cvcomputer{Domicilio:}{Avenida C\'ordoba 515}{Localidad:}{Las Acequias, C\'ordoba, Argentina}
	\begin{minipage}{9cm}
		\begin{itemize}
			\item Ciudad: Río Cuarto, Córdoba
			\item Fecha de Nacimiento: 14/10/1983
			\item Lugar de Nacimiento: Ciudad de Buenos Aires
		\end{itemize}
	\end{minipage}
	\begin{minipage}{9cm}
		%\begin{itemize}
		%\item Fecha de Nacimiento: 14/10/1983
		%\item Lugar de Nacimiento: Buenos Aires City
		%\end{itemize}
	\end{minipage}
	
	\section*{Actividades actuales}
	\cventry{}{}{}{}{}{Actualmente me desempeño como Ayudante de Primera con dedicación Simple en el Departamento de Computación de la Universidad Nacional de Río Cuarto. Mi área de investigación se centra en la calidad de software, haciendo incapié en generación y evaluación de tests; reparación automática de programas; y especificaciones formales.}
	
	\section*{Estudios Realizados}
	\cventry{2014-2019}{Doctorado en Ciencias de la Computaci\'on}{}{FAMAF, UNC}{Córdoba}{Tema de investigaci\'on: En esta tesis desarrollamos un nuevo operador de mutaci\'on para expresiones de navegaci\'on, el cual extiende al conjunto de operadores tradicionales en el \'ambito de mutation testing al permitir la representaci\'on de fallas no representadas anteriormente. Y en el \'ambito de reparaci\'on autom\'atica de programas, incrementa la reparabilidad de los mismos.}
	
	\cventry{2009-2014}{Licenciatura en Ciencias de la Computaci\'on}{}{UNRC}{Río Cuarto}{}
	
	\cventry{2007-2011}{Analista en Computaci\'on}{}{UNRC}{Río Cuarto}{}
	
	
	\section{Becas Obtenidas}
	\cventry{2019-2021}{Beca de Posdoctorado}{}{Otorgada por CONICET}{}{}
	\cventry{2014-2019}{Beca interna Doctoral}{}{Otorgada por CONICET}{}{}
	\cventry{2012}{Becas de Fin de Carrera para Estudiantes de Grado en Carreras TICs}{}{Otorgada por ANPCYT}{}{}
	\cventry{2011}{Beca de Estímulo a las Vocaciones Científicas.}{}{Otorgada por CIN}{}{}
	
	\section*{Participación en Proyectos de Investigación}
	
	\subsection*{Director/Co-Director}
	\cventry{2019}{Improving software testing through automatic translation of test inputs to API calls.}{Otorgado por Ministerio de Ciencia y Tecnología.}{}{}{Director.}
	
	\subsection*{Colaborador}
	\cventry{2016-2018}{Formal methods for Software Specification, Design, and Verification.}{PPI}{Otorgado por Secretaria de Ciencia y T\'ecnica, Universidad Nacional de R\'io Cuarto}{Project Director Dr. Nazareno Aguirre}{}
	
	\cventry{2012-2015}{Formal methods for Software Specification, Design, and Verification.}{PPI}{Otorgado por Secretaria de Ciencia y T\'ecnica, Universidad Nacional de R\'io Cuarto}{Project Director Dr. Nazareno Aguirre}{}
	
	\section*{Antecedentes Académicos}
	\subsection*{Posiciones Académicas}
	
	\cventry{Abril 2019- Actualidad}{Ayudante de Primera con dedicación Simple}{}{Departamento de Computación de la Universidad Nacional de Río Cuarto, Río Cuarto, Córdoba, Argentina.}{}{}
	
	\cventry{Abril 2018- Diciembre 2018}{Ayudante de Primera con dedicación Simple}{}{Departamento de Computación de la Universidad Nacional de Río Cuarto, Río Cuarto, Córdoba, Argentina.}{}{}
	
	\cventry{Agosto 2015- Marzo 2017}{Ayudante de Primera con dedicación Simple}{}{Departamento de Computación de la Universidad Nacional de Río Cuarto, Río Cuarto, Córdoba, Argentina.}{}{}
	
	\cventry{2008-2012}{Ayudante de Segunda con dedicación Simple}{}{Departamento de Computación de la Universidad Nacional de Río Cuarto, Río Cuarto, Córdoba, Argentina.}{}{}
	
	\section*{Publicaciones}
	
	\subsection*{Conferencias Internacionales}
	\cventry{2022}{ICEBAR: Feedback-Driven Iterative Repair of Alloy Specifications.}{Simón Gutiérrez Brida, Germán Regis, Guolong Zheng, Hamid Bagheri, ThanhVu Nguyen, Nazareno Aguirre, Marcelo F. Frias}{37th International Conference on Automated Software Engineering. (ASE 2022)}{Michigan, US}{Full Research Paper. (A1 Qualis)}
	
	\cventry{2022}{ATR: template-based repair for Alloy specifications.}{Guolong Zheng, ThanhVu Nguyen, Simón Gutiérrez Brida, Germán Regis, Marcelo F. Frias, Nazareno Aguirre, Hamid Bagheri}{31st International Symposium on Software Testing and Analysis (ISSTA 2022)}{Virtual, South Korea}{Full Research Paper. (A2 Qualis)}
	
	\cventry{2021}{FLACK: Localizing Faults in Alloy Models.}{Guolong Zheng, ThanhVu Nguyen, Simón Gutiérrez Brida, Germán Regis, Marcelo F. Frias, Nazareno Aguirre, Hamid Bagheri}{36th International Conference on Automated Software Engineering (ASE 2021)}{Melbourne, Australia}{Tool Paper. (A1 Qualis)}
	
	\cventry{2021}{BeAFix: An Automated Repair Tool for Faulty Alloy Models.}{Simón Gutiérrez Brida, Germán Regis, Guolong Zheng, Hamid Bagheri, ThanhVu Nguyen, Nazareno Aguirre, Marcelo F. Frias}{36th International Conference on Automated Software Engineering (ASE 2021)}{Melbourne, Australia}{Tool Paper. (A1 Qualis)}
	
	\cventry{2021}{Artifact of 'FLACK: Counterexample-Guided Fault Localization for Alloy Models'.}{Guolong Zheng, ThanhVu Nguyen, Simón Gutiérrez Brida, Germán Regis, Marcelo F. Frias, Nazareno Aguirre, Hamid Bagheri}{43rd International Conference on Software Engineering (ICSE 2021)}{Madrid, Spain}{Artifact Paper.}
	
	\cventry{2021}{FLACK: Counterexample-Guided Fault Localization for Alloy Models.}{Guolong Zheng, ThanhVu Nguyen, Simón Gutiérrez Brida, Germán Regis, Marcelo F. Frias, Nazareno Aguirre, Hamid Bagheri}{43rd International Conference on Software Engineering (ICSE 2021)}{Madrid, Spain}{Full Research Paper. (A1 Qualis)}
	
	\cventry{2021}{Artifact of Bounded Exhaustive Search of Alloy Specification Repairs.}{Guolong Zheng, ThanhVu Nguyen, Simón Gutiérrez Brida, Germán Regis, Marcelo F. Frias, Nazareno Aguirre, Hamid Bagheri}{43rd International Conference on Software Engineering (ICSE 2021)}{Madrid, Spain}{Artifact Paper.}
	
	\cventry{2021}{Bounded Exhaustive Search of Alloy Specification Repairs.}{Simón Gutiérrez Brida, Germán Regis, Guolong Zheng, Hamid Bagheri, ThanhVu Nguyen, Nazareno Aguirre, Marcelo F. Frias}{43rd International Conference on Software Engineering (ICSE 2021)}{Madrid, Spain}{Full Research Paper. (A1 Qualis)}
	
	\cventry{2019}{On the effect of object redundancy elimination in randomly testing collection classes.}{P. Ponzio, V. Bengolea, S. Guti\'errez Brida, G. Scilingo, N. Aguirre, M. Frias}{11th International Workshop on Search-Based Software Testing (SBST 2018)}{Gothenburg, Sweden}{Full Research Paper.}
	
	\cventry{2017}{An Analysis of the Suitability of Test-based Patch Acceptance Criteria}{L. Zem\'in, S. Guti\'errez, A. Godio, C. Cornejo, R. Degiovanni, G. Regis, N. Aguirre and M. Frias}{10th International Workshop on Search-Based Software Testing (SBST 2017)}{Buenos Aires, Argentina}{Full Research Paper.}
	
	\cventry{2017}{Boolean expression extender - A mutation operator for strengthening and weakening boolean expressions}{S. Guti\'errez Brida, G. Scilingo}{2017 XLIII Latin American Computer Conference (CLEI 2017)}{C\'ordoba, Argentina}{Full Research Paper. (B4 Qualis)}
	
	\cventry{2017}{DynAlloy Analyzer: A Tool for the Specification and Analysis of Alloy Models with Dynamic Behaviour}{G. Regis, C. Cornejo, S. Guti\'errez Brida, M. Politano, F. Raverta, P. Ponzio, N. Aguirre, J.P. Galeotti, M. Frias}{11th Joint Meeting of the European Software Engineering Conference and the ACM SIGSOFT Symposium on the Foundations of Software Engineering (ESEC/FSE 2017)}{Paderborn, Alemania}{Tool Paper. (A2 Qualis)}
	
	\subsection*{Conferencias Nacionales}
	
	\cventry{2019}{Detecci\'on de objetos espurios en generaci\'on autom\'atica de entradas}{Sim\'on Guti\'errez Brida, Pablo Ponzio, Valeria Bengolea, y Nazareno Aguirre}{en Congreso Argentino de Ciencias de la Computaci\'on 2019 (CACIC 2019)}{R\'io Cuarto, Argentina}{}
	
	\cventry{2019}{Implementaci\'on B\'asica de Typestates en Rust}{Marcelo Arroyo, Sim\'on Guti\'errez Brida, Pablo Ponzio}{en Congreso Argentino de Ciencias de la Computaci\'on 2019 (CACIC 2019)}{R\'io Cuarto, Argentina}{}
	
	\cventry{2012}{Criterios de Cobertura sobre RepOK para Reducir Test Suites Exhaustivas Acotadas: Estudio de Casos}{V. Bengolea, S. Guti\'errez Brida y N. Aguirre}{en Congreso Argentino de Ciencias de la Computaci\'on 2012 (CACIC 2012)}{Bah\'ia Blanca, Argentina}{}
	
	\section*{Evaluación de Artículos}
	\subsection*{Conferencias Internacionales}
	\cventry{2022}{FSE/ESEC 2022 Artifacts Track}{Evaluación de dos publicaciones de Artefactos}{}{}{}
	
	\subsection*{Conferencias Nacionales y Regionales}
	\cventry{2019}{SLISW 2019}{Evaluación de un artículo en Conferencia Latinoamericana de Informática}{}{}{}
	
	\cventry{2017}{SLISW 2017}{Evaluación de dos artículos en Conferencia Latinoamericana de Informática}{}{}{}
	
	\cventry{2017}{ASSE 2017}{Evaluación de dos artículos en Conferencia Latinoamericana de Informática}{}{}{}
	
	\section*{Supervisión de tesis y tesinas de grado}
	
	\cventry{2017 - 2019}{Leandro Buttignol.}{Comparison of Automatic test generation tools.}{}{Co-dirección con Dr. Pablo Ponzio (Director), y Dra. Valeria Bengolea (Co-Director)}{}{}
	
	\section*{Integrante de Comité Evaluador}
	
	\subsection*{Concursos Docentes}
	\cventry{2012}{Concurso de Ayudantes de Segunda del Departamento de Computación}{}{Miembro Titular del Comité Evaluador}{Universidad Nacional de Río Cuarto}{}
	
	\section*{Asistencia a Congresos y Jornadas Científicas}
	\cventry{2021}{ICSE 2021}{}{43rd International Conference on Software Engineering, Madrid, Spain}{}{}
	\cventry{2017}{ICSE 2017}{}{39th International Conference on Software Engineering, Buenos Aires, Argentina}{}{}
	\cventry{2014}{JAIIO/CLEI 2017}{}{46 Jornadas Argentinas de Inform\'atica / 43 Conferencia Latinoamericana de Inform\'atica, Universidad Tecnol\'ogica Nacional, C\'ordoba, Argentina}{}{}
	\cventry{2008-2019}{Escuela de Verano de Ciencias Inform\'aticas RIO}{}{ediciones 2008, 2009, 2010, 2011, 2012, 2013, 2015, 2016, 2017, 2018, 2019, y 2020}{Departamento de Computaci\'on, Universidad Nacional de Río Cuarto}{}
	
	\section*{Tareas Académicas de Gestión/Administrativas}
	
	\cventry{2021-2022}{Integrante Graduado Titular del Consejo Departamental}{}{en el Departamento de Computaci\'on de la Universidad Nacional de Río Cuarto}{}{Per\'iodo: a partir de Agosto de 2021 hasta Agosto de 2022.}
	
	\cventry{2019-2021}{Integrante Graduado Suplente del Consejo Departamental}{}{en el Departamento de Computaci\'on de la Universidad Nacional de Río Cuarto}{}{Per\'iodo: a partir de Octubre de 2019 hasta Agosto de 2021.}
	
	\cventry{2017-2019}{Integrante Graduado Titular del Consejo Departamental}{}{en el Departamento de Computaci\'on de la Universidad Nacional de Río Cuarto}{}{Per\'iodo: a partir de Diciembre de 2017 hasta Octubre de 2019.}
	
	\section*{Participación u Organización de Eventos Científico-Tecnológicos}
	
	\cventry{2016-2020}{Escuela de Verano de Ciencias Informáticas (Escuela RIO)}{Universidad Nacional de Río Cuarto}{Miembro del comité organizador}{}{}
	
	\cventry{2017}{ICSE 2017 39th International Conference on Software Engineering}{Buenos Aires, Argentina}{Estudiante Voluntario}{}{}
	
\end{document}