\documentclass[11pt,a4paper]{moderncv}

\usepackage{enumitem}
% moderncv themes
%\moderncvtheme[blue]{casual}                 % optional argument are 'blue' (default), 'orange', 'red', 'green', 'grey' and 'roman' (for roman fonts, instead of sans serif fonts)
\moderncvtheme[blue]{classic}                % idem

\usepackage[T1]{fontenc}
% character encoding
\usepackage[utf8x]{inputenx}                   % replace by the encoding you are using
\usepackage[english]{babel}

% adjust the page margins
\usepackage[scale=0.85]{geometry}
\recomputelengths                             % required when changes are made to page layout lengths

\fancyfoot{} % clear all footer fields
\fancyfoot[LE,RO]{\thepage}           % page number in "outer" position of footer line
\fancyfoot[RE,LO]{\footnotesize } % other info in "inner" position of footer line

%number is section
\newcounter{secnumber}
\renewcommand\sectionstyle[1]{{%
  \refstepcounter{secnumber}%
  \sectionfont
  \textcolor{color1}{\thesecnumber~#1}%
}}

% personal data
\firstname{\Huge Sim\'on Emmanuel}
\familyname{Guti\'errez Brida}
\title{Curriculum Vitae} 
%\address{<Address Street>}{<Addres Place>}
\mobile{+549 (358) 4396 444}
\phone{+549 (358) 4676 235}
%\fax{<Fax number>}
\email{sgutierrez@dc.exa.unrc.edu.ar}
%\extrainfo{additional information (optional)} % optional, remove the line if not wanted
%\photo[100pt]{pict.JPG}
%\photo[84pt]{foto.jpg}                         % '64pt' is the height the picture must be resized to and 'picture' is the name of the picture file; optional, remove the line if not wanted
%\quote{"Success is the ability to go from failure to failure without losing your enthusiasm." -- Winston Churchill}                 % optional, remove the line if not wanted

%\nopagenumbers{}                             % uncomment to suppress automatic page numbering for CVs longer than one page


%----------------------------------------------------------------------------------
%            content
%----------------------------------------------------------------------------------

\usepackage{amsmath}
\usepackage{amssymb}
\usepackage{latexsym}
\usepackage{wasysym}

\begin{document}
\maketitle

\section{Informaci\'on Personal}
\begin{minipage}{9cm}
\begin{itemize}
\item Nombre: Guti\'errez Brida, Sim\'on Emmanuel
\item D.N.I.: 30591131
\item Domicilio: Moreno 657 7D
\item Localidad: Río Cuarto, C\'ordoba
\end{itemize}
\end{minipage}
\begin{minipage}{9cm}
\begin{itemize}
\item Fecha de Nacimiento: 14/10/1983
\item Lugar de Nacimiento: Ciudad de Buenos Aires
\item Estado Civil: Soltero
\item Nº de Hijos: 0
\end{itemize}
\end{minipage}


\section{Estudios Realizados}
%\cventry{start-end}{<Position Held>}{<Name of employer>}{<Place>}{<Country>}{<Description>} % arguments 3 to 6 are optional
\cventry{2014-2019}{Doctorado en Ciencias de la Computaci\'on}{}{Facultad de Matem\'atica, Astronom\'ia y F\'isica, Universidad Nacional de C\'ordoba}{C\'ordoba, Argentina}{Tema de investigaci\'on: En esta tesis desarrollamos un nuevo operador de mutaci\'on para expresiones de navegaci\'on, el cual extiende al conjunto de operadores tradicionales en el \'ambito de mutation testing al permitir la representaci\'on de fallas no representadas anteriormente. Y en el \'ambito de reparaci\'on autom\'atica de programas, incrementa la reparabilidad de los mismos.}

\cventry{2009-2014}{Licenciatura en Ciencias de la Computaci\'on}{}{Universidad Nacional de Río Cuarto}{Río Cuarto, C\'ordoba, Argentina}{Promedio Final: 8.53}

\cventry{2007-2011}{Analista en Computaci\'on}{}{Universidad Nacional de Río Cuarto}{Río Cuarto, C\'ordoba, Argentina}{Promedio Final: 8.13}

%\section{Área de Investigación}
%Actualmente me desempeño como becario Post Doctoral de CONICET en el Departamento de Computación de la UNRC. Mi supervisor es el Dr. Nazareno Aguirre.
%En líneas generales, mi trabajo de investigación está vinculado a la Ingeniería de Software, más precisamente, a la calidad de software.
%Asegurar que un programa resuelve el problema correcto de la manera correcta, es de fundamental importancia práctica y económica en la actividad de producción de software. A su vez, especificar de manera correcta el problema a resolver y modelar la solución apropiada, es también de fundamental importancia.
%Concretamente, nuestro objetivo es la evaluación del software (testing y verificación automática de especificaciones formales), así como apoyar actividades asociadas a la reparación de software y especificaciones incorrectas (localización de fallas y reparación automática).


\section{Becas Obtenidas}
\cventry{2019-2021}{Beca de PostDoctorado}{}{otorgada por Consejo Nacional de Investigaciones Cient\'ificas y T\'ecnicas (CONICET)}{}{}
\cventry{2014-2019}{Beca interna Doctoral}{}{otorgada por Consejo Nacional de Investigaciones Cient\'ificas y T\'ecnicas (CONICET)}{}{}
\cventry{2012}{Becas de Fin de Carrera para Estudiantes de Grado en Carreras TICs}{}{otorgada por Agencia Nacional de Promoci\'on Cient\'ifica y Tecnol\'ogica}{}{}
\cventry{2011}{Beca de Est\'imulo a las Vocaciones Cient\'ificas.}{}{otorgada por Consejo Interuniversitario Nacional}{}{}

\section{Antecedentes Laborales}
\subsection{Acad\'emicos de Grado}

\cventry{Agosto 2015- Marzo 2022}{Ayudante de Primera Simple}{}{en el Departamento de Computaci\'on, Universidad Nacional de Río Cuarto}{afectado a las asignaturas \emph{Algoritmos 1} , \emph{Complejidad y Computabilidad}, \emph{Algoritmos 2}, \emph{Programación Avanzada}, y \emph{Ingenier\'ia de software distribuida y terciarizada}}{}

\cventry{2008-2012}{Ayudante de Segunda Rentado}{}{en el Departamento de Computaci\'on, Universidad Nacional de Río Cuarto}{afectado a las asignaturas \emph{Algoritmos 1}, \emph{Algoritmos 2}, y \emph{Introducci\'on a la Algor\'itmica y Programaci\'on}}{}


\subsection{Otras Actividades Docentes}
\cventry{Febrero 2010- Mayo 2010}{Tutor}{}{Proyecto PACENI. Fac. de Ciencias Exactas, Físico-Químicas y Naturales de la Universidad Nacional de Río Cuarto}{Resoluci\'on Decanal 16 del 2010}{}

\cventry{2010}{Colaborador en el dictado del M\'odulo Problem\'atica Universitaria y Sociedad y de tutor\'ia de aspirantes al ingreso 2010}{}{Fac. de Ciencias Exactas, Físico-Químicas y Naturales de la Universidad Nacional de Río Cuarto}{Resoluci\'on Decanal 17 del 2010}{}

\cventry{Abril 2009- Julio 2009}{Tutor}{}{Proyecto 2009 ``Tutor\'ias de pares''. Fac. de Ciencias Exactas, Físico-Químicas y Naturales de la Universidad Nacional de Río Cuarto}{afectado a la asignatura \emph{Introducci\'on a la Algor\'itmica y Programaci\'on}}{Resoluci\'on Decanal 491/09}

\cventry{Abril 2008- Julio 2008}{Tutor}{}{Proyecto 2008 ``Tutor\'ias de pares''. Fac. de Ciencias Exactas, Físico-Químicas y Naturales de la Universidad Nacional de Río Cuarto}{afectado a la asignatura \emph{Introducci\'on a la Algor\'itmica y Programaci\'on}}{}

\section{Participaci\'on en Proyectos de Investigaci\'on}
\subsection{Como Director/Investigador Responsable}

\cventry{2019- 2020}{Mejorando el testing de software mediante la reconstrucción automática de objetos a partir de APIs.}{Grupo de Reciente Formación con Tutor (GRFT)}{Ministerio de Ciencia y Tecnología, Gobierno de Córdoba}{}{Director}

\subsection{Como Integrante del Grupo Responsable}

\cventry{2020- 2022}{Desarrollo y aplicación de análisis estático para detección de defectos en software.}{Programa y Proyecto de Investigación (convocatoria PPI)}{Secretaría de Ciencia y Técnica, Universidad Nacional de Río Cuarto}{Director: Marcelo Arroyo}{Investigador}

\cventry{2016- 2018}{Análisis Automático de Programas basado en Satisfactibilidad Booleana.}{Programa y Proyecto de Investigación (convocatoria PPI)}{Secretaría de Ciencia y Técnica, Universidad Nacional de Río Cuarto}{Director: Nazareno Aguirre}{Estudiante}

\cventry{2012- 2015}{Métodos Formales para la Especificación, Diseño y Verificación de Software.}{Programa y Proyecto de Investigación (convocatoria PPI)}{Secretaría de Ciencia y Técnica, Universidad Nacional de Río Cuarto}{Director: Nazareno Aguirre}{Estudiante}

\section{Publicaciones}

\subsection{Conferencias Internacionales}
\cventry{2021}{FLACK: Localizing Faults in Alloy Models.}{Guolong Zheng, ThanhVu Nguyen, Simón Gutiérrez Brida, Germán Regis, Marcelo F. Frias, Nazareno Aguirre, Hamid Bagheri}{36th International Conference on Automated Software Engineering (ASE 2021)}{Melbourne, Australia}{Tool Paper.}

\cventry{2021}{BeAFix: An Automated Repair Tool for Faulty Alloy Models.}{Simón Gutiérrez Brida, Germán Regis, Guolong Zheng, Hamid Bagheri, ThanhVu Nguyen, Nazareno Aguirre, Marcelo F. Frias}{36th International Conference on Automated Software Engineering (ASE 2021)}{Melbourne, Australia}{Tool Paper.}

\cventry{2021}{Artifact of 'FLACK: Counterexample-Guided Fault Localization for Alloy Models'.}{Guolong Zheng, ThanhVu Nguyen, Simón Gutiérrez Brida, Germán Regis, Marcelo F. Frias, Nazareno Aguirre, Hamid Bagheri}{43rd International Conference on Software Engineering (ICSE 2021)}{Madrid, Spain}{Artifact Paper.}

\cventry{2021}{FLACK: Counterexample-Guided Fault Localization for Alloy Models.}{Simón Gutiérrez Brida, Germán Regis, Guolong Zheng, Hamid Bagheri, ThanhVu Nguyen, Nazareno Aguirre, Marcelo F. Frias}{43rd International Conference on Software Engineering (ICSE 2021)}{Madrid, Spain}{Full Research Paper.}

\cventry{2021}{Artifact of Bounded Exhaustive Search of Alloy Specification Repairs.}{Guolong Zheng, ThanhVu Nguyen, Simón Gutiérrez Brida, Germán Regis, Marcelo F. Frias, Nazareno Aguirre, Hamid Bagheri}{43rd International Conference on Software Engineering (ICSE 2021)}{Madrid, Spain}{Artifact Paper.}

\cventry{2021}{Bounded Exhaustive Search of Alloy Specification Repairs.}{Simón Gutiérrez Brida, Germán Regis, Guolong Zheng, Hamid Bagheri, ThanhVu Nguyen, Nazareno Aguirre, Marcelo F. Frias}{43rd International Conference on Software Engineering (ICSE 2021)}{Madrid, Spain}{Full Research Paper.}

\cventry{2019}{On the effect of object redundancy elimination in randomly testing collection classes.}{P. Ponzio, V. Bengolea, S. Guti\'errez Brida, G. Scilingo, N. Aguirre, M. Frias}{11th International Workshop on Search-Based Software Testing (SBST 2018)}{Gothenburg, Sweden}{Full Research Paper.}

\cventry{2017}{An Analysis of the Suitability of Test-based Patch Acceptance Criteria}{L. Zem\'in, S. Guti\'errez, A. Godio, C. Cornejo, R. Degiovanni, G. Regis, N. Aguirre and M. Frias}{10th International Workshop on Search-Based Software Testing (SBST 2017)}{Buenos Aires, Argentina}{Full Research Paper.}

\cventry{2017}{Boolean expression extender - A mutation operator for strengthening and weakening boolean expressions}{S. Guti\'errez Brida, G. Scilingo}{2017 XLIII Latin American Computer Conference (CLEI 2017)}{C\'ordoba, Argentina}{Full Research Paper.}

\cventry{2017}{DynAlloy Analyzer: A Tool for the Specification and Analysis of Alloy Models with Dynamic Behaviour}{G. Regis, C. Cornejo, S. Guti\'errez Brida, M. Politano, F. Raverta, P. Ponzio, N. Aguirre, J.P. Galeotti, M. Frias}{11th Joint Meeting of the European Software Engineering Conference and the ACM SIGSOFT Symposium on the Foundations of Software Engineering (FSE 2017)}{Paderborn, Alemania}{Tool Paper.}

\subsection{Conferencias Nacionales}

\cventry{2021}{Bounded Exhaustive Search of Alloy Specification Repairs}{Simón Gutiérrez Brida, Germán Regis, Guolong Zheng, Hamid Bagheri, ThanhVu Nguyen, Nazareno Aguirre, Marcelo F. Frias}{Argentine Symposium on Software Engineering (ASSE)}{}{Oral Communication}

\cventry{2021}{FLACK: Counterexample-Guided Fault Localization for Alloy Models}{Guolong Zheng, ThanhVu Nguyen, Simón Gutiérrez Brida, Germán Regis, Marcelo F. Frias, Nazareno Aguirre, Hamid Bagheri}{Argentine Symposium on Software Engineering (ASSE)}{}{Oral Communication}

\cventry{2019}{Detecci\'on de objetos espurios en generaci\'on autom\'atica de entradas}{Sim\'on Guti\'errez Brida, Pablo Ponzio, Valeria Bengolea, y Nazareno Aguirre}{en Congreso Argentino de Ciencias de la Computaci\'on 2019 (CACIC 2019)}{Río Cuarto, Argentina}{}

\cventry{2019}{Implementaci\'on B\'asica de Typestates en Rust}{Marcelo Arroyo, Sim\'on Guti\'errez Brida, Pablo Ponzio}{en Congreso Argentino de Ciencias de la Computaci\'on 2019 (CACIC 2019)}{Río Cuarto, Argentina}{}

\cventry{2012}{Criterios de Cobertura sobre RepOK para Reducir Test Suites Exhaustivas Acotadas: Estudio de Casos}{V. Bengolea, S. Guti\'errez Brida y N. Aguirre}{en Congreso Argentino de Ciencias de la Computaci\'on 2012 (CACIC 2012)}{Bah\'ia Blanca, Argentina}{}

\section{Evaluación de Artículos}

\subsection{Conferencias Nacionales/Regionales}
\cventry{2019}{SLISW 2019}{Evaluación de un art\'iculo en Conferencia Latinoamericana de Informática}{}{}{}

\cventry{2017}{SLISW 2017}{Evaluación de dos art\'iculos en Conferencia Latinoamericana de Informática}{}{}{}

\cventry{2017}{ASSE 2017}{Evaluación de dos art\'iculos en Conferencia Latinoamericana de Informática}{}{}{}

\section{Formación de Recursos Humanos}

\subsection{Licenciatura en Ciencias de la Computaci\'on (5 a\~nos + Trabajo Final)}
\cventry{2017 - 2018}{Comparaci\'on de herramientas de generaci\'on autom\'atica de tests}{}{Director: Pablo Ponzio, Co-Directores: Sim\'on Emmanuel Guti\'errez Brida}{Tesista: Leandro Buttignol}{Finalizado}

\section{Integrante de Comité Evaluador}

\subsection{Concursos}
\cventry{2012}{Concurso de Ayudantes de Segunda del Departamento de Computación}{}{Miembro Titular del Comité Evaluador}{Universidad Nacional de Río Cuarto}{}


\section{Capacitaci\'on Realizada}

\subsection{Materias de Post Grado}
\cventry{2018}{Seguridad inform\'atica}{}{Dictado por el Mg. Marcelo Arroyo, en la Universidad Nacional de Río Cuarto}{Nota obtenida 10}{}

\cventry{2018}{Testing de software }{}{Dictado por el Dr. Pablo Ponzio, en la Universidad Nacional de Río Cuarto}{Nota obtenida 9}{}

\cventry{2015}{Certificaci\'on de programas usando COQ}{}{Dictado por el Dr. Carlos Luna y el Licenciado en Ciencias de la Computaci\'on Dante Zanarini, en la Universidad Nacional de C\'ordoba}{Nota obtenida 10}{}

\cventry{2015}{Dise\~no de Programas Concurrentes}{}{Dictado por el Dr. Germ\'an Regis y el Dr Nazareno Aguirre, en la Universidad Nacional de C\'ordoba}{Nota obtenida 9}{}

\cventry{2015}{Probabilidad y cadenas Markovianas}{}{Dictado por el Dr. Marcelo Ruiz, en la Universidad Nacional de C\'ordoba}{Nota obtenida 10}{}

\subsection{Cursos Aprobados}

\cventry{2019}{Seminario de Ingeniería de Software}{}{Dictado por Profs. Martin Wirsing, Univ. Ludwig-Maximillians de Munich; Carlo Ghezzi, Polit\'ecnico de Mil\'an; Sebast\'ian Uchitel, Univ. De Buenos Aires; Tom Maibaum, Univ. De McMaster; Marcelo Fr\'ias, Instituto Tecnol\'ogico de Buenos Aires}{26 Escuela de Verano de Ciencias Inform\'aticas RIO 2019. Nota obtenida 10}{}

\cventry{2016}{Seminario de Ingeniería de Software}{}{Dictado por Prof. Sarfraz Khurshid, The University of Texas at Austin, USA}{23 Escuela de Verano de Ciencias Inform\'aticas RIO 2016. Nota obtenida 7}{}

\cventry{2015}{Automatic Software Repair}{}{Dictado por Prof. Martin Monperrus, INRIA, France}{29 Escuela de Ciencias Informáticas ECI 2015. Nota obtenida 10}{}

\cventry{2011}{Distributed and Outsourced Software Engineering}{Asignatura optativa de Licenciatura en Ciencias de la Computación, dictada dentro del marco del proyecto con el mismo nombre por la Universidad ETH en Zurich a cargo del Dr Bertrand Meyer}{Aprobado 10}{}{}

\cventry{2011}{An\'alisis autom\'atico de programas : de la teor\'ia a la pr\'actica}{}{Dictado por Dr Diego Garbervetsky y el Licenciado Guido de Caso, Universidad de Buenos Aires, Argentina}{18 Escuela de Verano de Ciencias Inform\'aticas RIO 2011. Nota Obtenida 6.50}{}

\cventry{2010}{Primeros Pasos en la Docencia Universitaria}{}{Dictado por Mg. Azucena Alija, Universidad de Río Cuarto, Argentina}{}{}

\cventry{2010}{Testing de software con m\'etodos formales}{}{Dictado por Dra Laura Brandan Briones, Universidad Nacional de C\'ordoba, Argentina}{17 Escuela de Verano de Ciencias Inform\'aticas RIO 2010. Nota Obtenida 7}{}

\cventry{2009}{Stringology}{}{Dictado por Dr.Yo\'an Pinz\'on, Universidad Nacional de Colombia}{16 Escuela de Verano de Ciencias Inform\'aticas RIO 2009. Nota obtenida 8}{}

\cventry{2009}{Ingl\'es con fines espec\'ificos para las Ciencias Exactas: desarrollo de habilidades de comprensi\'on y producci\'on oral del discurso acad\'emico (Modulo 3)}{}{Organizado por el Departamento de Computaci\'on de la Facultad de Ciencias Exactas F\'isico-Qu\'imicas y Naturales de la UNRC y Dictado por el New English Institute}{}{}

\cventry{2008}{Primeros Pasos en la Docencia Universitaria}{}{Dictado por Mg. Azucena Alija, Universidad de Río Cuarto, Argentina}{}{}


%\subsection{Cursos Asistidos}

%\cventry{2019}{Conceptos de Model Checking}{}{Dictado por Prof. Pedro D'Argenio, Universidad Nacional de C\'ordoba, Argentina}{26 Escuela de Verano de Ciencias Inform\'aticas RIO 2019.}{}

%\cventry{2019}{Auto-QA Automation: Software para testear Software}{}{Dictado por Prof. Juan Pablo Galeotti, Universidad de Buenos Aires, Argentina}{26 Escuela de Verano de Ciencias Inform\'aticas RIO 2019.}{}

%\cventry{2013}{Introducci\'on a la computaci\'on heterog\'enea}{}{Dictado por Dr. Nicolás Wolovick y Lic. Carlos Bederían, Universidad Nacional de C\'ordoba, Argentina}{20 Escuela de Verano de Ciencias Inform\'aticas RIO 2013.}{}

%\cventry{2013}{T\'ecnicas avanzadas para testing automatizado}{}{Dictado por Dr Marcelo Fr\'ias, Instituto Tecnol\'ogico de Buenos Aires, Argentina}{20 Escuela de Verano de Ciencias Inform\'aticas RIO 2013.}{}

%\cventry{2013}{Seminario de Computaci\'on, Introducci\'on a la plataforma Android y al desarrollo de aplicaciones}{}{Dictado por Mg Marcelo  Arroyo, Universidad Nacional de Río Cuarto, Argentina}{}{}

%\cventry{2012}{Matching Problems Under Preferences}{}{Dictado por Dr Julián Mestre, University of Sydney, Australia}{19 Escuela de Verano de Ciencias Inform\'aticas RIO 2012.}{}

%\cventry{2012}{Programaci\'on paralela en GPUs}{}{Dictado por Dra Mar\'ia Fabiana Picolli, Universidad de San Luis, Argentina}{19 Escuela de Verano de Ciencias Inform\'aticas RIO 2012.}{}

%\cventry{2010}{Modelado y simulaci\'on de sistemas din\'amicos con el formalismo DEVS}{}{Dictado por Dr Ernesto Kofman, Universidad de Rosario, Argentina}{17 Escuela de Verano de Ciencias Inform\'aticas RIO 2010.}{}



\subsection{Talleres y Seminarios Asistidos}

\cventry{2020}{Octavo Taller Argentino de Fundamentos para el An\'alisis y Construcci\'on Autom\'atica de Software FACAS 2020}{}{Río Cuarto, C\'ordoba, Argentina}{}{}

\cventry{2019}{Séptimo Taller Argentino de Fundamentos para el An\'alisis y Construcci\'on Autom\'atica de Software FACAS 2019}{}{La Falda, C\'ordoba, Argentina}{}{}

\cventry{2017}{Sexto Taller Argentino de Fundamentos para el An\'alisis y Construcci\'on Autom\'atica de Software FACAS 2018}{}{La Falda, C\'ordoba, Argentina}{}{}

\cventry{2017}{Quinto Taller Argentino de Fundamentos para el An\'alisis y Construcci\'on Autom\'atica de Software FACAS 2017}{}{Carcaraes, Alto delta del Parana, Santa Fe, Argentina}{}{}

\cventry{2016}{Cuarto Taller Argentino de Fundamentos para el An\'alisis y Construcci\'on Autom\'atica de Software FACAS 2016}{}{Carcaraes, Alto delta del Parana, Santa Fe, Argentina}{}{}


\subsection{Asistencia a Congresos y Jornadas}
\cventry{2021}{ICSE 2021}{}{43rd International Conference on Software Engineering, Madrid, Spain}{}{}
\cventry{2017}{ICSE 2017}{}{39th International Conference on Software Engineering, Buenos Aires, Argentina}{}{}
\cventry{2014}{JAIIO/CLEI 2017}{}{46 Jornadas Argentinas de Inform\'atica / 43 Conferencia Latinoamericana de Inform\'atica, Universidad Tecnol\'ogica Nacional, C\'ordoba, Argentina}{}{}
\cventry{2008-2019}{Escuela de Verano de Ciencias Inform\'aticas RIO}{}{ediciones 2008, 2009, 2010, 2011, 2012, 2013, 2015, 2016, 2017, 2018, 2019, y 2020}{Departamento de Computaci\'on, Universidad Nacional de Río Cuarto}{}

\section{Tareas Acad\'emicas de Gestión/Administrativas}

\subsection{Integrante de Consejos y Comisiones}
\cventry{2021-2022}{Integrante Graduado Titular del Consejo Departamental}{}{en el Departamento de Computaci\'on de la Universidad Nacional de Río Cuarto}{}{Per\'iodo: a partir de Agosto de 2021 hasta Agosto de 2022.}

\cventry{2019-2021}{Integrante Graduado Suplente del Consejo Departamental}{}{en el Departamento de Computaci\'on de la Universidad Nacional de Río Cuarto}{}{Per\'iodo: a partir de Octubre de 2019 hasta Agosto de 2021.}

\cventry{2017-2019}{Integrante Graduado Titular del Consejo Departamental}{}{en el Departamento de Computaci\'on de la Universidad Nacional de Río Cuarto}{}{Per\'iodo: a partir de Diciembre de 2017 hasta Octubre de 2019.}

\section{Participación u Organización de Eventos Científico-Tecnológicos}

\cventry{2016-2020}{Escuela de Verano de Ciencias Informáticas (Escuela RIO)}{Universidad Nacional de Río Cuarto}{Miembro del comité organizador}{}{}

\cventry{2017}{ICSE 2017 39th International Conference on Software Engineering}{Buenos Aires, Argentina}{Estudiante Voluntario}{}{}

%\section*{Conocimiento de Idiomas}
%\cventry{2001}{Instituto de Ingl\'es STEPS}{}{}{}{}
%\cventry{1992-1999}{Instituto de Ingl\'es STEPS}{}{}{}{}

\section*{Experiencia en Desarrollo de Software}

\cventry{Noviembre 2013- Abril 2014}{Desarrollador \textit{C++} en proyecto de procesamiento de im\'agenes, reconocimiento de patrones}{Yeti Media, California}{}{}{}

%\cventry{2014- 2020}{Desarrollador \texttt{Java} en \textit{muJava++}, una extensi\'on de \textit{muJava}, herramienta de mutation testing}{\url{https://github.com/saiema/MuJava}}{}{}{}

%\cventry{2021- Actualidad}{Desarrollador \texttt{Java} en \textit{BeAFix}, Bounded Exhaustive Alloy Fix, herramienta para reparación automática de modelos Alloy}{\url{https://github.com/saiema/BeAFix}}{}{}{}

\end{document}